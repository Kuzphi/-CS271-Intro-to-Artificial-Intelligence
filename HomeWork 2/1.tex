\normalfont\documentclass[letterpaper,11pt]{article}
\usepackage{amsmath, amsfonts,amssymb,latexsym}
\usepackage{fullpage}
\usepackage{parskip}
\usepackage{flexisym}
\usepackage{indentfirst}
\usepackage{graphicx}
\usepackage{algorithm2e}
% \usepackage{algorithm}
\usepackage{algorithmicx}
% \usepackage{algpseudocode}
\usepackage{amsmath}
\begin{document}
\setlength{\parindent}{2ex}
\newcommand{\header}{
	\noindent \fbox{
	\begin{minipage}{6.4in}
	\medskip
	\textbf{CS 271 -  Introduction to Artificial Intelligence} \hfill \textbf{Fall 2016} \\[1mm]
	\begin{center}
		{\Large HomeWork 2} \\[3mm]
	\end{center}
	  Name: \itshape{Liangjian Chen} \\
	  \textnormal{ID}: \itshape{\#52006933} \hfill \today
	\medskip
	\end{minipage}}
}
\bigskip
\header

\begin{enumerate}
\item[Problem 1]\par
	\begin{enumerate}
		\item[Step 1] Zerind\\
			\begin{tabular}{|c|c|c|c|}
				\hline
				Name & g & h & f\\
				\hline
				Arad & 75 & 366 & 441\\
				Oradea & 71 & 380 & 451\\
				\hline
			\end{tabular}
		\item[Step 2]  Arad\\
			\begin{tabular}{|c|c|c|c|}
				\hline
				Name & g & h & f\\
				\hline
				Sibiu & 215 & 366 & 441\\
				Timisoara& 193 & 329 & 522\\
				Oradea & 71 & 380 & 451\\
				\hline
			\end{tabular}
		\item[Step 3] 
			Sibiu\\
			\begin{tabular}{|c|c|c|c|}
				\hline
				Name & g & h & f\\
				\hline
				Fagaras & 214 & 176 & 390\\
				Rimnicu Vilcea &  295 & 193 & 488\\
				Timisoara & 193 &  329 & 522\\
				Oradea & 71 & 380 & 451\\
				\hline
			\end{tabular}
		\item[Step 4]
			Fagaras\\
			\begin{tabular}{|c|c|c|c|}
				\hline
				Name & g & h & f\\
				\hline
				Bucharest & 425 & 0 & 425\\
				Rimnicu Vilcea &  295 & 193 & 488\\
				Timisoara & 193 &  329 & 522\\
				Oradea & 71 & 380 & 451\\
				\hline
			\end{tabular}
		\item [Step 5]
			Find Bucharest, the distance is 25.
	\end{enumerate}
\item[Problem 2]\par
	Assume $u$ expand $v$ and $s$ is the start point, $t$ is the destination.\\
	\begin{enumerate}	
		\item
		\begin{flalign*}
			f(v) = c(s,v) + h(v) = c(s,u) + c(u,v) + h(v) \ge c(s,u) + h(u) = f(u)
		\end{flalign*}
		\item
		\begin{flalign*}
		&h_1(u) \le c(u,v) + h_1(v)\\
		&h_2(u) \le c(u,v) + h_2(v)\\
		&h = \max\{h_1(u),h_2(u)\} \le \max\{c(u,v) + h_1(v), c(u,v) + h_2(v)\}\\
		&= c(u,v) + \max\{h_1(v),h_2(v)\} = c(u,v) + h(v)
		\end{flalign*}
		\item prove it is admissible is proving the optimal path would always be found. Were it not he case, there would be another node $n^\prime$ in the frontier on the optimal path from $s$ to $t$. Let's say path 1 is the path find in heuristic search and path 2 is the optimal path go through $n^\prime$ and $d_1$ ,$d_2$ are their corresponding length. Thus,\\$f(u) = c(s,n^\prime) + h(n^\prime) \le c(s,n^\prime) + c(n^\prime,t) = d_1 \le d_2 = f(t)$\\
		Then, in the last step, the frontier should pop the $n^\prime$ rather than $t$. So it contradict with the assumption. Thus the statement holds.
		\item consider, after we pick $u$ out from the frontier, node $v$ link to $u$ again. From(a), we can obtain that $g(u) + h(u) \le g(v) + h(v)$ and $h(u) \ge h(v)$. The new weight is $f_{new}(u) = g(v) + c(v,s) + h(u) \ge g(u) + h(u)$. Thus, it would never update the $f(u)$ and would never push $u$ into frontier again.
		\item Intuitively, $h(u)$ is an underestimate of $c(u,t)$, so if $h_1(u) \ge h_2(u)$, $h_1(u)$ is more accurate than $h_2(u)$. Thus, $A_1^*$ would expand \textbf{equal or less} than $A_2^*$.\par
		For formal proof, let's consider picking node $u$ in the frontier by $h_2$, but there is another node $u^\prime$ whose $f$-value is smaller than $u$ by heuristic function $h_1$ which leads $u$. Then $A_1^*$ will choose $u^\prime$. If it leads to an optimal path without $u$, $A_1^*$ does not need to expand $u$ anymore. \par
		However if no case mentioned above happened, the number of expand node should be same.
	\end{enumerate}\par
\item [Problem 3]\par
Solution:
	\begin{enumerate}
		\item[Completeness:]
			For any $w$, algorithm will expand all the node and find the answer.\\
		\item[Optimal:]
		\begin{flalign*}
			&f(n) = (2 - w) g(n) + wh(n)\\
			&f(n) = (2 - w)(g(n) + \frac{w}{2 - w}h(n))\\
			&\text{the coefficient of $h(n)$ should less or equal $1$}\\
			&\frac{w}{2-w} \le 1\\
			&w \le 1\\
		\end{flalign*}
		\item[$w$ = 0:]
		Uniform-cost search
		\item[$w$ = 1:]
		Best-first search
		\item[$w$ = 2:]
		Greedy best search
	\end{enumerate}	
\item[Problem 4]
	Yes, we can use this result to cut some brunch. \par
	If a node $u$'s estimated distance $f(u)$ is larger than $f_U$, we can just discard it , rather than putting it into frontier. 
\item [Problem 5]\par
	For example, a graph $G(V,E)$, $V = \{s,t,a,b\}$, $E = \{e(s,a,0),e(s,b,0),e(a,t,100),e(b,t,101)\}$. Heuristic function $h(a) = 90, h(b) = 10$. So we would expand $b$ before $a$ and get goal node $t$, at this time $f(t) = 101$, but the optimal path is $s\to a\to t$ which just cost $100$. So terminating as soon as goal node find is not correct.


\end{enumerate}
\end{document}
