\normalfont\documentclass[letterpaper,11pt]{article}
\usepackage{amsmath, amsfonts,amssymb,latexsym}
\usepackage{fullpage}
\usepackage{parskip}
\usepackage{flexisym}
\usepackage{indentfirst}
\usepackage{graphicx}
\usepackage{algorithm2e}
% \usepackage{algorithm}
\usepackage{algorithmicx}
% \usepackage{algpseudocode}
\usepackage{amsmath}
\begin{document}
\setlength{\parindent}{2ex}
\newcommand{\header}{
	\noindent \fbox{
	\begin{minipage}{6.4in}
	\medskip
	\textbf{CS 271 -  Introduction to Artificial Intelligence} \hfill \textbf{Fall 2016} \\[1mm]
	\begin{center}
		{\Large HomeWork 6} \\[3mm]
	\end{center}
	  Name: \itshape{Liangjian Chen} \\
	  \textnormal{ID}: \itshape{\#52006933} \hfill \today
	\medskip
	\end{minipage}}
}
\bigskip
\header

\begin{enumerate}
\item[Problem 1]\textbf{Solution:}\par
	Define cost function as how many constrains are violated. At beginning the cost is $4$.
	\begin{enumerate}
		\item \par
			In original expression, $x$ and $y$ could be same. Thus it should be revised as following:
			$$\neg \exists x,y,n\text{ }Person(x) \land Person(y) \land \neg(x = y) \land HasSS\#(x,n) \land HasSS\#(y,n)$$ 
		\item \par
			Yes, it is correct.
		\item \par
			No. In original expression, it said that everyone has every different SSN, which obviously incorrect.
			$$\forall x,n\text{ }Person(x) \land HasSS\#(x,n) \Rightarrow Digits(n,9)$$
		\item \par
		Assume $SS\#(x)$ means $x$'s social security number.
		$$\neg \exists x,y\text{ }Person(x) \land Person(y) \land (SS\#(x) = SS\#(y))$$
		$$SS\#(Jhon) = SS\#(Mary)$$
		$$\forall x\text{ }Person(x) \Rightarrow Digits(SS\#(x),9)$$
	\end{enumerate}
\item[Problem 2]\textbf{Solution:}\par
	\begin{enumerate}
		\item No
		\item $x = A, y = B , z = B$
		\item $x = David, father(x) = George$
		\item $x = g(u) = g(f(v))$
		\item $x = y = z = B$
	\end{enumerate}
\item[Problem 3]\textbf{Solution:}\par
	\begin{align}
		&Alpine(Tony), Alpine(Mike), Alpine(John).\\
		&\forall x, Alpine(x)  \Rightarrow (skier(x)\land \neg climber(x)) \lor (\neg skier(x) \land climber(x))\\
		&\forall x, climber(x) \Rightarrow \neg like(x,Rain) \\
		&\forall x, skier(x) \Rightarrow like(x,snow) \\
		&\forall x, like(Jhon, x) \Rightarrow \neg like(Mike, x)\\
		&\forall x, \neg like(Jhon, x) \Rightarrow like(Mike, x)\\
		&\neg like(John, rain)\\
		&\neg like(John, snow)
		\text{from (6) (7) (8), we can obtain}
		like(mike,rain)
	\end{align}
\item[Problem 4]\textbf{Solution:}\par
\item[Problem 5]\textbf{Solution:}\par
\item[Problem 6]\textbf{Solution:}\par
\item[Problem 7]\textbf{Solution:}\par
\item[Problem 8]\textbf{Solution:}\par
\end{enumerate}
\end{document}
