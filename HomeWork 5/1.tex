\normalfont\documentclass[letterpaper,11pt]{article}
\usepackage{amsmath, amsfonts,amssymb,latexsym}
\usepackage{fullpage}
\usepackage{parskip}
\usepackage{flexisym}
\usepackage{indentfirst}
\usepackage{graphicx}
\usepackage{algorithm2e}
% \usepackage{algorithm}
\usepackage{algorithmicx}
% \usepackage{algpseudocode}
\usepackage{amsmath}
\begin{document}
\setlength{\parindent}{2ex}
\newcommand{\header}{
	\noindent \fbox{
	\begin{minipage}{6.4in}
	\medskip
	\textbf{CS 271 -  Introduction to Artificial Intelligence} \hfill \textbf{Fall 2016} \\[1mm]
	\begin{center}
		{\Large HomeWork 5} \\[3mm]
	\end{center}
	  Name: \itshape{Liangjian Chen} \\
	  \textnormal{ID}: \itshape{\#52006933} \hfill \today
	\medskip
	\end{minipage}}
}
\bigskip
\header

\begin{enumerate}
\item[Problem 1]\textbf{Solution:}\par
	Define cost function as how many constrains are violated. At beginning the cost is $4$.
	\begin{enumerate}
		\item \par
			answer is 1.
			Only $B$ = True, $A$ = True.
		\item \par
			answer is 15.
			$A$ = False or $B$ = False or $C$ = False or $D$ = False;
		\item\par
			answer is 0.
	\end{enumerate}
\item[Problem 2]\textbf{Solution:}\par
	\begin{enumerate}
		\item[Set:]
			Locate(a,b): a is at b' house.			
		\item[Statement:]
			$\neg \text{locate(car,Fred)} \to \text{locate(a,John)}$
		\item[Constrains:]
			No, I can not.
	\end{enumerate}
\item [Problem 3]\textbf{Solution:}\par
First, it is obvious that $a$ must be true, otherwise for any logical expression $s$ $s \land \neg a = Flase$ i.e. it is unsatisfied.\par
\begin{tabular}{c|c|c|c|c}
			\hline
			$a$ & $P$ & $Q$ & $P\lor a$ & $Q \lor \neg a$   \\
			\hline
			1 & 0 & 0 & 1 & 0 \\
			1 & 0 & 1 & 1 & 1 \\
			1 & 1 & 0 & 1 & 0 \\
			1 & 1 & 1 & 1 & 1 \\
		\end{tabular}\\
		From the table, we can conclude, for any logical expression $P,Q$, $P\lor a = 1, Q\lor \neg a = Q$. Thus the unit resolution is sound.
\item[Problem 4]\textbf{Solution:}\par

\begin{align*}
	\neg[((P\lor \neg Q)\to R)\to(P\land Q)]\\
	\neg[(\neg(P\lor \neg Q)\lor R)\to(P\land Q)]\\
	\neg[\neg(\neg(P\lor \neg Q)\lor R)\lor(P\land Q)]\\
	(\neg P\lor R)\land (Q \lor R) \land (\neg P \lor \neg Q)
\end{align*}
\item[Problem 5]\textbf{Solution:}\par
		$r_{i,j} = \neg q_{i,1}\land \neg q_{i,2}\land ... \land \neg q_{i,j-1} \land q_{i,j} \land \neg q_{i,j+1}\land...\land \neg q_{i,n}$\\
		$R_i = r_{i,1} \lor r_{i,2} \lor ... \lor r_{i,n}$\\
		$Row = R_0 \land R_1 \land ... \land R_n$\\

		$c_{i,j} = \neg q_{1,j}\land \neg q_{2,j}\land ... \land \neg q_{i-1,j} \land q_{i,j} \land \neg q_{i+1,j}\land...\land \neg q_{n,j}$\\
		$C_i = c_{1,i} \lor c_{2,i} \lor ... \lor c_{n,i}$\\
		$Col = C_1 \land C_2 \land ... \land C_n$\\

		$xd^1_{i,j} = \neg q_{1,j-i+1}\land \neg q_{2,j}\land ... \land \neg q_{i-1,j-1} \land q_{i,j} \land \neg q_{i+1,j+1}\land...\land \neg q_{n,i-j+m} (i\le j)$\\
		$xd^1_{i,j} = \neg q_{i-j+ 1,1}\land \neg q_{2,j}\land ... \land \neg q_{i-1,j-1} \land q_{i,j} \land \neg q_{i+1,j+1}\land...\land \neg q_{n,j -i + n}(i > j)$\\
		$xD_i = \bigvee  \limits_{x - y = i} xd_{x,y}$\\
		$xDiagnol = xD_{-m} \land xD_{-m+1} \land ... \land xD_n$\\

		$yd_{i,j} = \neg q_{1,i+j-1}\land \neg q_{2,i+j-2}\land ... \land \neg q_{i-1,j+1} \land q_{i,j} \land \neg q_{i+1,j-1}\land...\land \neg q_{i+j-1,1} (i\le j)$\\
		$yd_{i,j} = \neg q_{i+j-m,m}\land \neg q_{i+j-m+1,m - 1}\land ... \land \neg q_{i-1,j+1} \land q_{i,j} \land \neg q_{i+1,j-1}\land...\land \neg q_{n,m - i - j}(i > j)$\\
		$yD_i = \bigvee \limits_{x + y = i} d_{x,y}$\\
		$yDiagnol = D_{2} \land D_{3} \land ... \land D_{n+m}$\\


		Final answer is $Row \land Col \land xDiagnol \land yDiagnol$.
\item[Problem 6]\textbf{Solution:}\par
	\begin{enumerate}
		\item 
		Table is shown below:\par
		\begin{tabular}{c|c|c|c|c}
			\hline
			$P$ & $Q$ & $R$ & $P\land(Q\land R)$ & $(P\land Q)\land R$ \\
			\hline
			0 & 0 & 0 & 0 & 0 \\
			0 & 0 & 1 & 0 & 0 \\
			0 & 1 & 0 & 0 & 0 \\
			0 & 1 & 1 & 0 & 0 \\
			1 & 0 & 0 & 0 & 0 \\
			1 & 0 & 1 & 0 & 0 \\
			1 & 1 & 0 & 0 & 0 \\
			1 & 1 & 1 & 1 & 1 \\
		\end{tabular}
		\item Table is shown below:\par
		\begin{tabular}{c|c|c|c|c}
			\hline
			$P$ & $Q$ & $R$ & $P\land(Q\lor R)$ & $(P\land Q)\lor(P\land R)$ \\
			\hline
			0 & 0 & 0 & 0 & 0 \\
			0 & 0 & 1 & 0 & 0 \\
			0 & 1 & 0 & 0 & 0 \\
			0 & 1 & 1 & 0 & 0 \\
			1 & 0 & 0 & 0 & 0 \\
			1 & 0 & 1 & 1 & 1 \\
			1 & 1 & 0 & 1 & 1 \\
			1 & 1 & 1 & 1 & 1 \\
		\end{tabular}
		\item Table is shown below:\par	
		\begin{tabular}{c|c|c|c}
			\hline
			$P$ & $Q$ & $\neg (P\land Q)$ & $ \neg P \lor \neg Q$ \\
			\hline
			0 & 0 & 1 & 1 \\
			0 & 1 & 1 & 1 \\
			1 & 0 & 1 & 1 \\
			1 & 1 & 0 & 0 \\
		\end{tabular}

		\item Table is shown below:\par
		\begin{tabular}{c|c|c|c}
			\hline
			$P$ & $Q$ & $P \leftrightarrow Q$ & $ (P \land Q)\lor(\neg P \land \neg Q)$ \\
			\hline
			0 & 0 & 1 & 1 \\
			0 & 1 & 0 & 0 \\
			1 & 0 & 0 & 0 \\
			1 & 1 & 1 & 1 \\
		\end{tabular}
	\end{enumerate}
\item[Problem 7]\textbf{Solution:}\par
	\begin{enumerate}
		\item valid, $\neg Smoke \lor Smoke$.
		\item Neither
		\item Neither
			$\neg(\neg Smoke \lor Fire)\lor(Smoke\lor \neg Fire) \\ (\ Smoke \land \neg Fire)\lor Smoke\lor \neg Fire\\Smoke\lor \neg Fire$\\
		 
		\item valid, $Smoke \lor Fire \lor \neg Fire = Smoke \lor True = True$
		\item valid, $\neg Smoke \lor \neg Heat \lor Fire \leftrightarrow \neg Smoke \lor \neg Heat \lor Fire$. Left part and right part are same.
		\item valid $Big \lor Dumb \lor \neg Dumb \lor Big = True \lor Big = True$.
	\end{enumerate}
\item[Problem 8]\textbf{Solution:}\par
	$\{\neg P \lor Q, \neg L \lor \neg M \lor P, \neg B \lor \neg L \lor M, \neg A\lor \neg P \lor L,\neg A \lor \neg B \lor L, A, B\}$\\
	\begin{enumerate}
		\item[step 1] 
			pure symbol $Q = True$\\			
			$\{\neg L \lor \neg M \lor P, \neg B \lor \neg L \lor M, \neg A\lor \neg P \lor L,\neg A \lor \neg B \lor L, A, B\}$\\
		\item[step 2]
			unit clause $A = Ture$, $B = True$.\\
			$\{\neg L \lor \neg M \lor P, \neg L \lor M, \neg P \lor L, L\}$\\
		\item[step 3]
			unit clause $L = Ture$\\
			$\{\neg M \lor P, M\}$\\
		\item[step 3]
			pure symbol $P = True$\\
			$\{M\}$\\
		\item[step 3]
			unit clause $M = Ture$	\\
		Problem solved.
	\end{enumerate}
\end{enumerate}
\end{document}
